\documentclass[12pt]{article}
\usepackage{graphicx}
\usepackage[margin=2cm]{geometry}

\title{A Continuous Model Of Behavioural Phenotype: An Example Of
  Interaction Between \textit{Pieris brasicae} And
  \textit{Trichogramma} Wasps}

\author{Kelly Black\thanks{Coresponding Author, kjblack@gmail.con},
  \and Malcolm R Adams, \and Aladeen Al Basheer, \and Caner Kazanci,
  \and Bernard Patten, \and Stuart Whipple, \and Sofya Zaytseva}

\begin{document}

\maketitle

\section{Introduction}

There is great interest in the role of variation in animal
personalities\cite{doi:10.1111/j.1461-0248.2010.01536.x}.  The
variation in animal behaviours as well as the variation in the
coresponding genetic basis for animal personalities plays a role in
the broader population dynamics of the species as well as the other
animals that interact with the species. A variety of mathematical
models are being developed to explore this important aspect of
interactions. The models include compartment models, stochastic, agent
based models, as well as combinations of these approaches. One
difficulty, though, with such approaches is the challenge of
developing appropriate analytic tools to examine the general
properties of the resulting models.

We propose an approach that assumes a continuous distribution
associated with one behavioural aspect of an animal's behaviour. In
this initial exploration of a novel approach a relatively
straight-forward interaction is chosen. The example chosen is based on
the interactions between parasitic wasps, \textit{Trichogramma} Wasps,
and the butterfly \textit{Pieris
  brasicae}\cite{10.1093/beheco/arq007}.  The propensity of
\textit{Pieris brasicae} to employ a chemical associated with mating
behaviour is modeled as a continuous distribution, and a distributed
model is developed to approximate the resulting interactions.

In this treatment we approximate the system as consisting of two
species. Finding and parasitizing the butterflies' eggs requires a
relatively long interaction time, and a predator-prey relationship is
assumed. In particular, the predation rates are approximated using a
type II response. Due to the variation in behaviour, though, we adapt
the standard Holling response function for this new situation.

We first provide a more detailed overview of behavioural phenotypes
including some previous efforts to model the phenomena. Next, the
model used to provide insight into the phenomena is developed
including a re-derivation of a type II response assuming a continuous
variation in one behaviour. Following the derivation of the model, the
numerical approximation to the distributed system is briefly
discussed. Finally, some results from our numerical explorations are
provided.

\section{Behavioural Phenotype}
\label{section:behaviouralPhenotype}

The primary motivation for why the distribution of how animals behave
and interact is given in this section. We begin with a discussion of
the basic idea of behavioural phenotype. Next, a brief overview of
existing modeling efforts is provided. Finally a specific example of
social spiders is discussed, and in this example the phenomena is
directly observed and provides context for the importance of the
general phenomena.

A number of authors have noted the importance of the role of variation
in a species' habits or
genetics\cite{doi:10.1111/j.1461-0248.2010.01536.x,doi:10.1086/687235,mierzejewski_horn_luong_2019,SANTICCHIA20191}. We
begin this discussion with the survey provided by Kortet, \textit{et
  al}\cite{doi:10.1111/j.1461-0248.2010.01536.x}. The authors note
that within any large group of animals certain behaviours can vary
across the population. They go on to propose that the diversity of
behaviours can profoundly impact the broader dynamics of the
population of the animals as well as animals with a shared
interdependence.

Kortet, \textit{et al}\cite{doi:10.1111/j.1461-0248.2010.01536.x} note
the impact of variation in the distribution of behaviour. They note
some potential challenges this recognition implies, as well. One
challenge is the difficulty to identify and quantifyi the distribution
of animal personalities in a given population. A specific example
relates to the capture of Pumpkin Seed
Fish\cite{doi:10.1037/0735-7036.107.3.250}, and the methods used to
capture individuals in a population can result in a biased estimate.
Some methods may be more or less likely to capture a subset of
individuals based on the individuals' behaviours.

% https://www.semanticscholar.org/paper/Fish-behavioral-types-and-their-ecological-Mittelbach-Ballew/a381cb0b64a8107c694454e940fd7972263a8b77
% https://www.semanticscholar.org/paper/Are-most-samples-of-animals-systematically-biased-Biro/72eaa11f60d3b211f34ab04ff95211307978223b

(Find example from their paper.)  Other mathematical models have been
developed.... Compartmental models by citation. Other agent based
models by citation. Hybrid models by citation.


stochastic compartment model - Keeling and Grenfel

Stochastic model by Loyd

Compartment model by Loyd and Smith

A mathematical model examining the interactions within a colony of
social spiders, \textit{Stegodyphus dumicola}, is provided by
Pinter-Wollman, \textit{et al}\cite{doi:10.1086/687235}. The focus of
their work is in examining the impact of different boldness levels of
a key individual.  They note that a small number of individuals within
a colony may exhibit different levels of aggression, and a small
number of individuals can have a disproportionate impact in the
broader success of the colony.

As a means to model this important aspect of a colony, Pinter-Wollman,
\textit{et al}\cite{doi:10.1086/687235} provide a computational model
consisting of an agent based model with a stochastically generated
population. Their approach is based on a small number of rules that
result in a distribution of behaviours. The temporal interactions
within the colony are approximated via discrete time steps and
stochastic decision making. The external interactions are approximated
using an ordinary differential equation model.

In the resulting statistical analyses, Pinter-Wollman, \textit{et
  al}\cite{doi:10.1086/687235} found that different levels of boldness
within key individuals resulted in a difference in the mean boldness
levels of a broader group. For example, they found the differences in
boldness impacted the rate of capture of prey. They also found that
the rules themselves also impacted the disease dynamics within the
population.


\section{Modeling Genetic Variance As a Continuous Distribution}

Rather than construct a compartmental model or an agent based model, a
continuous, distributed parameter model is proposed. The resulting
model for the example system is a coupled partial differential
equation and an ordinary differential equation. Prior to the
derivation itself an overview of the specific system, the interactions
between butterflies and a parasitic wasp, are discussed. Next the
model is derived. Finally, and analysis of a simplified system of
ordinary differential equations is examined as a way to gain some
preliminary insights into the resulting system.

\subsection{Butterflies v. Wasps}
\label{butterflyVWasps}

The example of social spiders was briefly discussed earlier, but the
model developed here focuses on the interaction between butterflies,
\textit{Pieris brasicae}, and \textit{Trichogramma} wasps that prey on
the butterflies' eggs. The interactions between these populations has
received a good deal of attention
\cite{PMC2797620,doi:10.1111/j.1439-0418.1986.tb00834.x,Figueroa2010AttractionOT,10.3389/fpls.2019.01768}. We
focus on the results of one source in particular Huigens \textit{et
  al}\cite{10.1093/beheco/arq007}. In their work, the authors state
that the pressure placed on the butterfly population due to the wasps'
interactions results in a change in the long term behaviour of the
butterfly population.

The source of this pressure is the nature of the mating practices of
\textit{Pieris brasicae}. In order to attract a mate, the female
butterflies tend to emit a pheromone designed to attract the males to
the females. After copulation, though, it is not in either the
female's nor the male's best interest to continue to attract other
butterflies. In response, the male butterflies have a propensity to
apply another pheromone, referred to as an anti-aphrodisiac, that will
reduce the effectiveness of the original pheromone emitted by the
female.

Some species of \textit{Trichogramma} wasps parasitize the eggs of the
butterflies. Sometimes they physically attach themselves to a
butterfly and ride it until the butterfly begins to attach their eggs
in a preferable location. The wasps are able to detect the presence of
the anti-aphrodisiac which is correlated with an increased probability
that a butterfly with the anti-aphrodisiac is more likely to lay
eggs. In response, the wasps are more likely to ride on a butterfly if
it detects the anti-aphrodisiac. It is important to note this method
of predation on a butterfly's eggs requires a non-trivial time
commitment with respect to handling and detecting the presence of the
eggs.

The authors\cite{10.1093/beheco/arq007} note that the probability that
a male will make use of the anti-aphrodisiac varies between
individuals. There are different trade-offs in the use of the
anti-aphrodisiac, but the authors found that the wasps' actions are
applying direct evolutionary pressure with respect to the habits of
the butterflies. In particular they provide evidence that the
distribution of the propensity of the male butterflies to make use of
the anti-aphrodisiac is changing in time, and over time a growing
number of butterflies are less likely to engage in the behaviour.

\subsection{Modeling Behavioural Phenotype As a Continuous
  Distribution}

As mentioned in section \ref{section:behaviouralPhenotype}, a number
of options have been employed to model a distribution of behavioural
preferences in a given population. These methods include compartmental
models, agent based models, and hybrids of these two models. One
potential issue, though, is that the number of variations can be quite
high. For example (citation genetic sites - Abolins and Abols)
examines a system in which over 120 genetic sites determine a given
animal's propensity. The resulting number of combinations of such
sites can be extremely high.

Rather than treat the possible behaviours as a discrete distribution
we instead treat it as a continuous distribution. The way in which
this manifests itself within a model is to treat the population as a
distribution that depends on a new parameter. This new independent
variable is introduced within the model, and the parameter is now a
function that will vary as the independent variable varies.

At first glance, this is akin to a Bayesian statistical approach that
is often used to estimate the value of a parameter. (cite Ramsay,
Fitzpatrick) In these approaches a probability distribution is assumed
to describe the likelihood that a parameter takes on a given range of
values. In our case, though, we turn this around and assume that the
population itself varies, and the relative population density depends
on the underlying variable.

A question arises about how to translate the impact of this parameter
within a given model. In this initial examination of the approach we
employ the simplest option and assume a linear impact with respect to
the new independent variable. In particular, for the interactions
between the butterflies and wasps we track the population densities of
the two populations. Due to the time scale of interactions between the
butterflies and wasps as well as direct mortality of the interaction
we assume a predator-prey relationship rather than a disease like
relationship modeled in other parasitic systems.

The system is adapted from Tewa, \textit{et al}\cite{TEWA20134825}. We
first focus on the model as a system of ordinary differential
equations and later adapt it for the distributed system described
above. The basic assumption is that in the absence of predation the
butterfly population will act like a logistic equation. In the absence
of any butterflies the wasps will slowly die out. For this particular
system \textit{brasicae} is a generalist predator, and this is an
assumption that should be revisited beyond this initial
investigation. Finally, it is assumed that the interaction between the
two can be approximated as a type II response\cite{TEWA20134825}, and
the resulting ordinary differential equation is given by
\begin{eqnarray}
  \label{eq:initialSystem1}
  \frac{d}{dt} b(t) & = & \alpha \cdot b(t) (K - b(t)) - \gamma \cdot w(t) \frac{b(t)}{c+b(t)}, \\
  \label{eq:initialSystem2}
  \frac{d}{dt} w(t) & = & -d \cdot w(t) + g \cdot w(t) \frac{b(t)}{c+b(t)}.
\end{eqnarray}
The density of butterflies is modeled by the function $b(t)$, the
density of wasps is modeled by the function $w(t)$, and the parameters
$\alpha$, $\gamma$, $d$, and $g$ are positive constants. 

For this situation, though, there is a distribution associated with
the butterflies' propensity to make use of the anti-pheromone. We do
not model the male and female populations separately, but treat them
as a single population that intermingles in a relatively uniform
fashion. The propensity for a butterfly to make use of the
anti-pheromone depends on a new parameter, $\theta$. The value of
$\theta$ is assume to vary between $0$ and some positive constant,
$L$, and the larger the value of $\theta$ the more likely an
individual is to make use of the anti-pheromone. The distribution of
the population of butterflies is now dependent on both the time and
the new parameter,
\begin{eqnarray}
  b & = & b(t,\theta).
\end{eqnarray}

Our first task is to determine how the type II, or Holling type
response, should be expressed in this new context. Formal derivations
of the response function for the case given in equations
(\ref{eq:initialSystem1}) and (\ref{eq:initialSystem2}) are provided
bye Dawes and Sousa\cite{DAWES201311}.  In the derivation here,
however, we follow the approach in Holling's original
discussion\cite{holling_1959A,holling_1959B}. We begin by finding an
expression that relates the number of eggs parasitized by a single
wasp,
\begin{eqnarray}
  \label{eq:processingTime}
  Y(\theta) & = & r T_{\mathrm{search}}(\theta) b(t,\theta) p(\theta),
\end{eqnarray}
where $T_{\mathrm{search}}$ is the search time required by the wasp to
find the location of a butterfly's clutch. The motivation for this is
that the higher the value of $\theta$ the higher the success rate for
a given wasp.  The time a wasp requires to parasitize the clutch is
assumed to be proportional to the number of eggs, and the total time
is
\begin{eqnarray}
  \label{eq:totalEgglayingTime}
  T_{\mathrm{total}}(\theta) & = & T_{\mathrm{search}}(\theta) + \beta Y(\theta).
\end{eqnarray}
Substituting the search time found in equation
(\ref{eq:totalEgglayingTime}) back into equation
(\ref{eq:processingTime}) results in an expression for the rate of
predation per wasp,
\begin{eqnarray}
  \label{eq:waspPredationRate}
  Y(\theta) & = & \frac{r\beta T_{\mathrm{total}} b(t,\theta) p(\theta)}{1 + \beta r b(t,\theta) p(\theta)}.
\end{eqnarray}

In this case $p(\theta)$ represents the impact of the use of the
anti-pheromone, and we assume a linear relationship with a positive
slope. As $\theta$ increases the more likely an individual butterfly
is to make use of the anti-pheromone and the more likely a wasp is to
locate and ride along with a butterfly. The function, $p(\theta)$,
will be used to balance the positive and negative impacts in the
growth as well as the predation terms in equation
(\ref{eq:initialSystem1}), and the parameters $\alpha$ and $\gamma$
provide relative scales for the two impacts. We will normalize
$p(\theta)$ because it will be scaled later.

First, $p(\theta)$ should be a strictly positive function. If it is
zero then equation (\ref{eq:initialSystem1}) will indicate no growth
and no predation. We can then assume $p(0)$ is a minimal value,
$p(0)=a$, and the relationship has a positive slope. The resulting
normalized form is
\begin{eqnarray}
  \label{eq:linearFormP}
  p(\theta) & = & a + \frac{2(1-aL)}{L^2} \theta.
\end{eqnarray}
where the slope is chosen so that the area under the line from
$\theta=0$ to $\theta=L$ is one. One immediate restriction on the
parameters requires that $0<aL<1$. The interpretation of the slope is
that the greater the slope the greater the variation in the observed
trait.

Turning back to equation (\ref{eq:processingTime}), the rate of
predation is solved for, and the result is
\begin{eqnarray}
  \label{eq:rateOfPredation}
  r & = &  \frac{p(\theta) b(t,\theta) }{1 + \beta p(\theta) b(t,\theta)}.
\end{eqnarray}
This result can be immediately substituted into equation
(\ref{eq:initialSystem1}) in the more traditional form as
\begin{eqnarray}
  \label{eq:butterflyPredationRate}
  \gamma \cdot w(t) \frac{p(\theta) b(t,\theta) }{c +  p(\theta) b(t,\theta)}
\end{eqnarray}
Equation (\ref{eq:initialSystem2}), though, is for the rate of predation
for the single wasp population, so the impact for the wasps must be
accumulated across the entire butterfly population,
\begin{eqnarray}
  \label{eq:totalWaspPredationRate}
  \int^L_{\theta=0} g \cdot w(t) \frac{p(\theta) b(t,\theta) }{c + p(\theta) b(t,\theta)} ~ d\theta.
\end{eqnarray}

We are almost ready to put these terms together for the current
model. It is assumed that the mixing and interactions within the
butterfly population is uniform, and the diffusion of the trait
roughly follows Fick's law\cite{logan2006applied}. Here, a second
order derivative term approximates the sharing of the genetic
information relative to the propensity to use the
anti-aphrodisiac. The result is a coupled PDE and ODE system given by
\begin{eqnarray}
  \label{eq:odePDE1}
  \frac{\partial}{\partial t} b(t,\theta) & = &
      \alpha \cdot p(\theta) b(t,\theta) (K - b(t,\theta))
      - \gamma \cdot w(t) \frac{p(\theta) b(t,\theta)}{c+p(\theta)b(t,\theta)}
      + \mu \frac{\partial^2}{\partial \theta^2} b(t,\theta) , \\
  \label{eq:odePDE2}
  \frac{d}{dt} w(t) & = & -d \cdot w(t) +
      \int^L_{\theta=0} g \cdot w(t) \frac{p(\theta) b(t,\theta) }{c + p(\theta) b(t,\theta)} ~ d\theta.
\end{eqnarray}

The variables can be scaled as $b\rightarrow \bar{B}\hat{b}$,
$w\rightarrow \bar{W}\hat{w}$, $t\rightarrow \bar{T}\hat{t}$, and
$\theta\rightarrow \bar{\Theta}\hat{\theta}$. We choose $\bar{B}=K$,
$\bar{W}=\frac{\alpha K^2 a}{\gamma}$, $\bar{T}=\frac{1}{\alpha K a}$,
and $\bar{\Theta}=L$. The resulting system can be reduced to the
following form
\begin{eqnarray}
  \label{eq:scaledodePDE1}
  \frac{\partial}{\partial t} b & = &
      \hat{p}(\theta) b (1 - b)
      -  w \frac{\hat{p}(\theta) b}{c+\hat{p}(\theta)b}
      + \mu \frac{\partial^2}{\partial \theta^2} b , \\
  \label{eq:scaledodePDE2}
  \frac{d}{dt} w & = & -d \cdot w +
      \int^1_{\theta=0} g \cdot w \frac{\hat{p}(\theta) b }{c + \hat{p}(\theta) b} ~ d\theta,
\end{eqnarray}
where
\begin{eqnarray}
  \hat{p}(\theta) & = & 1 + m \cdot \theta.
\end{eqnarray}
The domain for the scaled independent variable, $\theta$, is
$0\leq\theta\leq 1$. As previously noted, the slope, $m$, can be
interpreted to indicate how much variation in the trait is present in
a population. The larger $m$ is the greater the amount of variation.
With respect to the PDE, Neumann boundary conditions are used in the
subsequent numerical approximations below. A variety of initial
conditions have been explored, and they are described in the results
section, section \ref{section:results}.


\subsection{Stability Analysis Of a Simplified System Of ODEs}

Before proceeding to the numerical approximation of the full model, we
first examine the stability of a simplified system. The simplified
system is found by examining an ODE with a similar form. By assuming
that the distribution of the butterflies is a constant consistent with
a single value of $\theta$, the system is approximated by the
following system of ODEs:
\begin{eqnarray}
  \label{eq:scaledODE1}
  \frac{d}{dt} b(t) & = &
      \hat{p}(\theta) b(t) (1 - b(t))
      -  w(t) \frac{\hat{p}(\theta) b(t)}{c+\hat{p}(\theta)b(t)}, \\
  \label{eq:scaledODE2}
  \frac{d}{dt} w(t) & = & -d \cdot w(t) +
       g \cdot w(t) \frac{\hat{p}(\theta) b(t) }{c + \hat{p}(\theta) b(t)}.
\end{eqnarray}

by focusing on the system of ODEs, we can examine an approximation to
the system by treating the variable $\theta$ as a fixed parameter and
can gain some insight into the distributed system. The analysis here
focused on the stability of the linearized system, and we focus on the
non-trivial fixed points that occur in the first quadrant away from
$b=0$ and away from $w=0$.

We first determine the nullclines for the system.
The $b$-nullclines for the system in the $b-w$ plane are given by
\begin{eqnarray}
  \label{eq:bnullclines}
  b & = & 0, \\
  w & = & (1-b)(c+b\cdot(1+m\theta)).
\end{eqnarray}
The second $b$-nullcline is a parabola opening downward, and the vertex is located at
$b=\frac{1+m\theta-c}{2(1+m\theta)}$. The $w$-nullclines are given by
\begin{eqnarray}
  \label{eq:wnullclines}
  w & = & 0, \\
  b & = & \frac{cd}{(g-d)(1+m\theta)}.
\end{eqnarray}
The second $w$-nullcline is a vertical line in the $b-w$ plane. In
order for a fixed point to exist in the first quadrant an immediate
restriction on the parameters requires that $g>d$.  Given the
nullclines, the only fixed point that occurs in the first quadrant
away from an axis is at $b=\frac{cd}{(1+m\theta)(g-d)}$ and
$w=\left(\frac{cd}{g-d}+c\right)
\left(1-\frac{cd}{(1+m\theta)(g-d)}\right)$.

The Jacobian for the system in equations (\ref{eq:scaledODE1}) and
(\ref{eq:scaledODE2}) at the fixed point is
\begin{eqnarray}
  J & = &
          \left[
          \begin{array}{rr}
            J_{11} & J_{12} \\
            J_{21} & 0
          \end{array}
          \right],
\end{eqnarray}
where
\begin{eqnarray}
  \label{eq:jacobian}
  J_{11} & = & (1+m\theta)(1-2b) - w (1+m\theta)\frac{c}{(c+(1+m\theta)b)^2}, \\
  J_{12} & = & \frac{-b\cdot(1+m\theta)}{c+(1+m\theta)b}, \\
  J_{21} & = & \frac{g\cdot c \cdot w \cdot (1+m\theta)}{(c+(1+m\theta)b)^2}.
\end{eqnarray}
The eigen values of the Jacobian have negative real part when the
trace is negative,
\begin{eqnarray}
  (1+m\theta)(1-2b) - w (1+m\theta)\frac{c}{(c+(1+m\theta)b)^2} & < & 0.
\end{eqnarray}
Substituting the values of $b$ and $w$ for the non-trivial fixed point
yields
\begin{eqnarray}
  \label{eqn:traceNegative}
  \frac{(1+m\theta)-c}{2(1+m\theta)} & < & \frac{cd}{(g-d)(1+m\theta)}.
\end{eqnarray}
This expression can be viewed through another substitution. Let
$u=1+m\theta$ and $v=\frac{cd}{g-d}=\frac{c}{\frac{g}{d}-1}$, and the
relationship in equation (\ref{eqn:traceNegative}) is now
\begin{eqnarray}
  \label{eq:stabilityParameters}
  v & > & \frac{u-c}{2}.
\end{eqnarray}
Note also that under this transformation, the value of the butterfly
density at the fixed point is $\frac{v}{u}$. The scaling in the
original model is based on the carrying capacity which presumes that
\begin{eqnarray}
  \frac{v}{u} & < & 1,
\end{eqnarray}
or 
\begin{eqnarray}
  \label{eq:boundFixedPoint}
  v & < & u.
\end{eqnarray}

A feasible region that implies stability of the fixed point can be
determined in the $u-v$ plane. If the parameters result in a point in
the region bounded by equation (\ref{eq:stabilityParameters}),
equation (\ref{eq:boundFixedPoint}), and $v>0$ then the resulting
fixed point will be linearly stable. A graphical depiction of the
stability region in the $u-v$ plane is shown in Figure
\ref{fig:uvStabilityRegion}.

\begin{figure}[htb]
  \centering
  \includegraphics[width=10cm]{odeStability-uv-plane.pdf}
  \caption[Stability region in the $u-v$ plane.]{Graphical view of the
    combined feasible and stability region in the $u-v$ plane.}
  \label{fig:uvStabilityRegion}
\end{figure}

In the context of the full, distributed system in equation
(\ref{eq:scaledodePDE1}) and equation (\ref{eq:scaledodePDE2}) the
domain of $\theta$ is $0\leq\theta\leq 1$. For a given set of
parameters the value of $v$ is fully determined. The values of $u$
vary from $u=1$ to $u=1+m\theta$.  The corresponding region in the
$u-v$ plane is along a horizontal line segment from $(1,v)$ to
$(1+m\theta,v)$ as shown in Figure \ref{fig:distributedLineSegment}.

\begin{figure}[htb]
  \centering
  \includegraphics[width=12cm]{odeStability-uv-plane-Line.pdf}
  \caption[Domain of the distributed system in the $u-v$
  plane.]{Domain of the parameters for the distributed system for a
    given set of parameters for $0\leq\theta\leq 1$. The dotted line
    represents the values of $\hat{p}(\theta)=1+m\theta$ for all
    values of $\theta$. The figure on the left is an example of a
    smaller value of $m$ while the figure on the right is an example
    of a larger value of $m$. The whole population shown in the figure
    on the left lie in a stable region, while part of the population
    shown in the figure on the right lie in an unstable region.}
  \label{fig:distributedLineSegment}
\end{figure}

\begin{figure}[htb]
  \centering
  \includegraphics[width=12cm]{ODEButterflyBounds.pdf}
  \caption[Upper and lower bounds of the butterfly density.]{The upper
    and lower bounds of the long term density of the butterfly
    population. The population was approximated for different values
    of $m$. The approximation was stopped if it either became close to
    a steady state or a limit cycle.}
  \label{fig:odeButterflyBifurcation}
\end{figure}


\section{Numerical Approximation}
\label{numericalApproximation}

We introduce the method for the numerical approximation of the coupled
PDE and ODE system in equations (\ref{eq:scaledodePDE1}) and
(\ref{eq:scaledodePDE2}). For the full system, we make use of a
Legendre pseudo-spectral collocation
method\cite{spectralMethodsFluids,hesthaven_gottlieb_gottlieb_2007,gottlieb1977numerical}. First,
the butterfly density is discretized in $\theta$ as
\begin{eqnarray}
  \label{eqn:spatialDiscretization}
  b_N(t,\theta) & = & \sum^N_{i=0} \hat{b}_i(t) \phi_i(\theta),
\end{eqnarray}
where $\phi_i(\theta)$ is the Lagrange interpolant on the
$i$\textsuperscript{th} abscissa of the Legendre-Gauss-Lobatto
quadrature\cite{hesthaven_gottlieb_gottlieb_2007}.

The abscissa of the Legendre-Gauss-Lobatto quadrature are the zeroes
of the function
\begin{eqnarray}
  \psi(\theta) & = & \left(1-\theta^2\right) L_{N}'(\theta),
\end{eqnarray}
where $L_N(\theta)$ is the $n$\textsuperscript{th} Legendre
polynomial\cite{davis2007methods}.  The weights, $w_n$, associated
with the quadrature are identical to those described in
Davis\cite{davis2007methods} as well as Golub and
Welsch\cite{gaussQuadratureRules}. In our examples the quadrature is
approximated using the methods described by Golub and
Welsch\cite{gaussQuadratureRules}.  Given $\psi(\theta)$ the Lagrange
interpolants can be calculated
\begin{eqnarray}
  \phi_i(\theta) & = & \frac{\psi(\theta)}{\psi'(\theta)(\theta-\theta_i)}.
\end{eqnarray}
The function $\psi(\theta)$ coincides with the Legendre differential
equation, and the relationship can be simplified as
\begin{eqnarray}
  \psi'(\theta) & = & -N(N+1)L_N(\theta).
\end{eqnarray}

The discretization of equations (\ref{eq:scaledodePDE1}) and
(\ref{eq:scaledodePDE2}) is constructed by substituting the definition
of $b_N$ from equation (\ref{eqn:spatialDiscretization}). A Galerkin
approximation is constructed, and one minor complication is that the
Lagrange interpolants are polynomials of degree $N$, and the
Gauss-Lobatto quadrature is exact for polynomials up to degree
$N$. The integral is approximated using the Gauss-Lobatto quadrature,
and the resulting sum results in an equivalent norm compared to the
Gauss quadrature which is exact for polynomials up to degree
$N$\cite{SobolevCanutoQuarteroni}.

The resulting variational form for Neumann boundary conditions using a
Galerkin approximation and a discrete inner product for equation
(\ref{eq:scaledodePDE1}) is
\begin{eqnarray}
  \sum_{j=0}^N \sum_{i=0}^N  \hat{b}_i'(t) \phi_i(\theta_j) \phi_k(\theta_j) w_j
  & = &
  \sum_{j=0}^N \sum_{i=0}^N \hat{p}(\theta_j)  \hat{b}_i(t) (1 - \hat{b}_i(t) ) \phi_i(\theta_j) \phi_k(\theta_j) w_j \\
  & &  -  \sum_{j=0}^N w(t) \frac{\hat{p}(\theta_j) \sum_{i=0}^N \hat{b}_i(t) \phi_i(\theta_j) }{c+\hat{p}(\theta_j) \sum_{i=0}^N \hat{b}_i(t) \phi_i(\theta_j)} \phi_k(\theta_j) w_j \nonumber \\ 
  & & - \sum_{j=0}^N \mu  \sum_{i=0}^N \hat{b}_i(t) \phi_i'(\theta_j) \phi_k'(\theta_j)  w_j, \nonumber
\end{eqnarray}
for all integer values of $k$ from $0$ to $N$ inclusive.  The basis
functions are the Lagrange interpolants on the abscissa, and the
equations can be reduced to
\begin{eqnarray}
  \hat{b}_k'(t) 
  & = &
        \hat{p}(\theta_j) \hat{b}_k(t) (1 - \hat{b}_k(t) )
        -  w(t) \frac{\hat{p}(\theta_k) \hat{b}_k(t)) }{c+\hat{p}(\theta_k)  \hat{b}_k(t) }  
   - \frac{\mu}{w_k} \sum_{i=0}^N \hat{b}_i(t) \left( \sum_{j=0}^N  \phi_i'(\theta_j) \phi_k'(\theta_j)  w_j \right).
\end{eqnarray}

Combined with equation (\ref{eq:scaledodePDE2}) the approximation
consists of a system of $N+2$ ordinary differential equations. The
integral in equation (\ref{eq:scaledodePDE2}) is approximated using
the Legendre-Gauss-Lobatto quadrature. The temporal discretization of
the resulting system is constructed using a second order, implicit
multi-step scheme. The scheme is implemented as a fully implicit
second order Adams-Moulton scheme\cite{ascher2011first}. At each time
step the resulting non-linear system is approximated using Newton's
method.

\section{Results}
\label{section:results}

A collection of specific approximations is provided. We start with an
examination of the behaviour of the system of ODEs given in equations
(\ref{eq:scaledODE1}) and (\ref{eq:scaledODE2}). Next approximations
of the coupled PDE and ODE from equations (\ref{eq:scaledodePDE1}) and
(\ref{eq:scaledodePDE2}) are provided. In both sets of examples a
numerical exploration of the existance of a stable steady state or
limit cycles is given.

\subsection{Approximation of the system of ODEs}
\label{subsection:odeApproximation}

We first examine the behaviour of the system of ODEs given in
equations (\ref{eq:scaledODE1}) and (\ref{eq:scaledODE2}). Our focus
is on the stability characteristics of the fixed points as well as the
long term dynamics of the system. The behaviour at the non-trivial
fixed point is explored with respect to changes in the parameter
$m$. Because the change in $\theta$ is not relevant for the ODE we
assume that $\theta=1$.

An example of the long term behaviour is provided in Figure
\ref{fig:odeButterflyBifurcation}. The system of ODEs given in
equations (\ref{eq:scaledODE1}) and (\ref{eq:scaledODE2}) was
approximated using a Runga-Kutta Fehlberg method with a minimum step
size of 1.0E-5.  The value of the parameters in the example are
$c=2.8$, $g=0.6$, $d=0.1$. Approximations were constructed for values
of $m$ sarting at $m=0.1$ and ending at $m=15.0$.

The approximation for the initial value of $m$ was started near the
non-trivial steady state, but the values of the butterflies and wasps
were taken to be 95\% of the values for the steady state. The
approximation was run until it detected the state was not changing or
a limit cycle was reached. In this case, if both densities did not
change more than 1.0E-7 from a previous high or low value for over
2000 time steps it was considered to be at a steady state. For the
limit cycle if the minimum and maximum for both densities was within
1.0E-7 for two full cycles then it was considered to be in a limit
cycle.

After the initial approximation, the initial condition for the next
value of $m$ was the end state of the previous approximation. This
process was done for the values of $m$ starting at the lowest value
and increasing. the process was then repeated for the value of $m$
starting at the highest value and decreasing. The results were nearly
identical, and no hysteresis was detected.

The results of the series of approximations are shown in Figure
\ref{fig:odeButterflyBifurcation}. The values of $m$ are given along
the horizontal axis, and the values of the butterfly densities are
given along the vertical axis. The values of the maximum and the
minimum values of the butterfly densities are plotted after the
approximation was terminated.

For values of $m$ less than about 2.841 the system moved close to a
steady state. For larger values of $m$ the system approached a limit
cycle. (Need to check to see if this is consistent with the theory!)

\subsection{Approximation of the coupled PDE and ODE}

The numerical exploration of the full, coupled system of the PDE and
ODE, equations (\ref{eq:scaledodePDE1}) and (\ref{eq:scaledodePDE2}),
is examined. The approximation method is that same as described in
section \ref{numericalApproximation}. The parameter range is similar
to that in subsection \ref{subsection:odeApproximation}.

The value of the parameters in the example are $c=2.8$, $g=0.6$,
$d=0.1$, and $\mu=0.01$. The temporal behaviour of the system is
similar to that found in the previous subsection, subsection
\ref{subsection:odeApproximation}. For example, an approximation for
--- provide example for $m=5$. --- Need to discuss this.

The onset of a limit cycle is delayed with respect to the system of
ODEs. For this value of $m$ the fixed point for the ODEs results in a
limit cycle. However, in the case of the coupled PDE and ODE system a
limit cycle did not appear until much larger values of $m$. For
example, at $m=12$, a repeating patter was found. The results of an
approximation are shown in Figure \ref{fig:approximationM12Mu01}.

The results shown in Figure \ref{fig:approximationM12Mu01} are only
for a short time span in order to make it easier to show the change in
time. The initial condition in this case is a Gaussian with a center
at the right end point.  Similar patterns emerge for an initial
condition that is a Gaussian centered at the left, a constant initial
condition, and an initial condition with the butterfly density set at
the theoretical steady state given each value of $\theta$.


Approximations were constructed for
values of $m$ sarting at $m=0.1$ and ending at $m=15.0$.


\begin{figure}[htb]
  \centering
  \includegraphics[width=12cm]{approximation-mu-01-m-12.pdf}
  \caption[Approximation with $m=12$ and $\mu=0.01$.]{Approximation of
    the system, $m=12$, $\mu=0.01$, $c=2.8$, $g=0.6$, and $d=0.1$. }
  \label{fig:approximationM12Mu01}
\end{figure}

\begin{figure}[htb]
  \centering
  \includegraphics[width=12cm]{approximation-mu-01-m-15.pdf}
  \caption[Approximation with $m=15$ and $\mu=0.01$.]{Approximation of
    the system, $m=15$, $\mu=0.01$, $c=2.8$, $g=0.6$, and $d=0.1$. }
  \label{fig:approximationM15Mu01}
\end{figure}

\begin{figure}[htb]
  \centering
  \includegraphics[width=12cm]{maxMinByM-mu-01-04.pdf}
  \caption[Maximum and minimum values of the butterfly
  density]{Maximum and minimum values of the butterfly densities after
    a long time period as the parameter $m$ varies.}
  \label{fig:maxMinButterflySmallMu}
\end{figure}

\section{Conclusion}

\section{Acknowledgements}

\clearpage
\bibliographystyle{siam}
\bibliography{animalBehaviour}

\end{document}
