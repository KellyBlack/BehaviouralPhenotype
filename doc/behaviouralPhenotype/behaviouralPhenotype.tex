\documentclass[12pt]{article}
\usepackage{graphicx}
\usepackage[margin=2cm]{geometry}

\title{A Continuous Model Of Behavioural Phenotype: An Example Of
  Interaction Between \textit{Pieris Brasicae} and
  \textit{Trichogramma} Wasps}

\author{Kelly Black\thanks{Coresponding Author, kjblack@gmail.con},
  \and Malcolm R Adams, \and Aladeen Al Basheer, \and Caner Kazanci,
  \and Bernard Patten, \and Stuart Whipple, \and Sofya Zaytseva}

\begin{document}

\maketitle

\section{Introduction}

The variation in animal behaviours as well as the variation in
genetics for a given species plays a role in the broader population
dynamics of the species as well as the other animals that interact
with the species. A variety of mathematical models have been developed
to explore this important aspect of interactions. The models include
compartment models, stochastic, agent based models, as well as
combinations of these approaches. One difficulty with such approaches
is the lack of analytic tools to examine the general properties of the
resulting models.

We propose, instead, an approach assuming a continuous distribution
associated with one behavioural aspect of an animal's behaviour. The
example chosen is based on the interactions beteen parasitic wasps,
\textit{Trichogramma} Wasps, and the butterfly \textit{Pieris
  Brasicae} (cite Huigens).  The propensity of \textit{Pieris
  Brasicae} to employ a chemical associated with mating behaviour is
modeled as a continuous distribution, and a distributed model is
developed to approximate the resulting interactions.

In this treatment we approximate the system as consisting of two
species. The parastic wasp requires a relatively long interaction
time, and a predator-prey relationship is assumed. In particular, the
predation rates are approximated using a type II response. Due to the
variation in behaviour, though, we adapt the standard Holling response
function for this situation.

We first provide a more detailed overview of behavioural phenotypes
including some previous efforts to model the phenomena. Next, the
model used to provide insight into the phenomena is developed
including a re-derivation of a type II response assuming a continuous
variation in one behaviour. Following the derivation of the model, the
numerical approximation to the distributed system is briefly
discussed. Finally, some results from our numerical explorations are
provided.

\section{Behavioural Phenotype}
\label{section:behaviouralPhenotype}

The primary motivation for why the distribution of how animals behave
and interact is given here. We begin with a discussion of the basic
idea of behavioural phenotype. Next, a brief overview of existing
modeling efforts is provided. Finally a specific example of social
spiders is discussed, and in this example the phenomena is directly
observed and provides context for the magnitude of the phenomena.

A number of authors have noted the importance of the role of variation
in a species' genetics or habits (cite Pinter-Wollman, Kortet,
Keiser). We begin this discussion with the survey provided by Kortet,
\textit{et al} (cite). The authors note that within any large group of
animals certain behaviours can vary across the population. They go on
to propose that the diversity of behaviours can profoundly impact the
broader dynamics of the population of the animals as well as animals
with a shared interdependence.

Kortet, \textit{et al} (cite) note the impact of variation in the
distribution of behaviour, they also note potential challenges this
recognition implies. One difficulty is identifying and quantifying the
distribution. An example they provide is the methods used to capture
individuals in a population can result in a biased estimate. The
reason for this is that some methods may be more or less likely to
capture a subset of individuals based on how bold or timid the animals
may be.

(Find example from their paper.)  Other mathematical models have been
developed.... Compartmental models by citation. Other agent based
models by citation. Hybrid models by citation.

Another, more detailed example of the importance of behavioural
differences between individuals in a population is provided by Keiser
\textit{et al} (cite). In their work, the differences in levels of
aggression in a social spider, \textit{Stegodyphus dumicola} are
examined. Experiments in sampling individuals found significant
differences in boldness levels. They also found that the variation of
boldness levels within a group impacted the spread of microbes across
the broader population.

A mathematical model for this system is provided by Pinter-Wollman,
\textit{et al} (cite). The focus of their work is in examining the
impact of different boldness levels of a key individual.  They note
that a small number of individuals within a colony may exhibit vastly
different levels of aggression, and a small number of individuals can
have a disproportionate impact in the broader success of the colony.

As a means to model this important aspect of a colony, Pinter-Wollman,
\textit{et al} provide a computational model consisting of an agent
based model with a stochastically generated population. Their approach
is based on a small number of rules that result in a distribution of
behaviours. The temporal interactions within the colony are
approximated via discrete time steps and stochastic decision
making. The external interactions are approximated using an
ordinary differential equation model.

In the resulting statistical analyses, Pinter-Wollman, \textit{et al}
found that different levels of boldness within key individuals
resulted in a difference in the mean boldness levels of a broader
group. For example, they found the differences in boldness impacted
the rate of capture of prey. They also found that the rules themselves
impacted also impacted the disease dynamics within the population.


\section{Modeling Genetic Variance As A Continuous Distribution}

Rather than construct a compartmental model or an agent based model, a
continuous, distributed parameter model is proposed here. The
resulting model for the example system is a coupled partial
differential equation and an ordinary differential equation. Prior to
the derivation itself an overview of the specific system, the
interactions between butterflies and a parasitic wasp, are
discussed. Next the model is derived. Finally, and analysis of a
simplified system of ordinary differential equations is examined as a
way to gain some preliminary insights into the resulting system.

\subsection{Butterflies v. Wasps}

The example of social spiders was briefly discussed earlier, but the
model developed here focuses on the interaction of butterflies,
\textit{Pieris Brasicae}, and \textit{Trichogramma} wasps that prey on
the butterflies' eggs. The interactions between these populations has
received a good deal of attention (cite many). We focus on the results
of one source in particular (cite Huigens). In the work presented by
(Huigens), the authors state that the pressure placed on the butterfly
population due to the wasps interaction results in a change in the
long term behaviour of the butterfly population.

The source of this pressure is the nature of the mating practices of
\textit{Pieris Brasicae}. In order to attract a mate, the female
butterflies tend to emit a pheromone designed to attract the males to
the females. After copulation, though, it is not in either the
female's nor the male's best interest to continue to attract other
butterflies. In response, the male butterflies have a propensity to
apply another pheromone, referred to as an anti-aphrodisiac, that will
dissuade reduce the effectiveness of the original pheromone emitted by
the female.

Some species of \textit{Trichogramma} wasps parasitize the eggs of the
butterflies. They do so by physically attaching themselves to the
butterfly and riding until the butterfly begins to attach their eggs
to a preferable plant. The wasps are able to detect the presence of
the anti-aphrodisiac and are able to recognize the increased
probability that a butterfly with the anti-aphrodisiac is more likely
to lay eggs. In response, the wasps are more likely to ride on a
butterfly if it detects the anti-aphrodisiac. It is important to note
this method of predation on a butterfly's eggs requires a non-trivial
time commitment with respect to handling and detecting the presence of
the eggs.

The authors (cite Huigens) note that the probability that a male will
make use of the anti-aphrodisiac varies between individuals. There are
different trade-offs in the use of the anti-aphrodisiac, but the
authors found that the wasps actions are applying direct pressure to
the habits of the butterflies. In particular they provide evidence
that the distribution of the propensity of the male butterflies to
make use of the anti-aphrodisiac is changing in time, and over time a
growing number of butterflies are less likely to engage in the
behaviour.

\subsection{Modeling Behavioural Phenotype As A Continuous
  Distribution}

As mentioned in section \ref{section:behaviouralPhenotype}, a number
of options have been employed to model a behavioural distribution in a
given population. These methods include compartmental models, agent
based models, and hybrids of these two models. One potential issue,
though, is that the number of variations can be quite high. For
example (citation genetic sites) examines a system in which over 120
genetic sites determine a given animal's propensity. The number of
combinations of such sites can be extremely high.

In our treatment we instead tread the distribution as being
continuous. The way in which this manifests itself within a model is
to treat the population as a distribution that depends on a new
parameter. The parameter within the model will be allowed to vary
according to some function of the new parameter.

On the surface this is akin to a Bayesian statistical approach that is
often used to estimate the value of a parameter. (cite Bayesian
papers) In these approaches a probability distribution is assumed to
describe the likelihood that a parameter takes on a given range of
values. In our case, though, we turn this around and assume that the
population itself varies, and the relative population density depends
on the parameter.

The question then arises how to translate the impact of this parameter
into a given model. In this initial examination of the approach we
employ the simplest option and assume a linear impact of the given
parameter. In particular, for the interactions between the butterflies
and wasps we track the population densities of the two
populations. Due to the long time scale of interactions between the
butterflies and wasps we assume a predator-prey relationship rather
than a disease like relationship often modeled in other parasitic
systems.

The particular model examined is similar to that examined by (cite
stability of predator prey with type II response). A type II response
is assumed due to the aforementioned long time scale of
interaction. We assume logistic growth for the butterfly
population. In this case the wasps are a generalist predator, but in
this initial treatment we assume a single dependence with respect to
the wasp as a first step in modeling this system.

The system adapted from (cite type II paper) is begins with the system
of ordinary differential equations
\begin{eqnarray}
  \label{eq:initialSystem1}
  \frac{d}{dt} b(t) & = & \alpha \cdot b(t) (K - b(t)) - \beta \cdot w(t) \frac{b(t)}{c+b(t)}, \\
  \label{eq:initialSystem2}
  \frac{d}{dt} w(t) & = & -d \cdot w(t) + g \cdot w(t) \frac{b}{c+b(t)}.
\end{eqnarray}
The density of butterflies is modeled by the function $b(t)$, the
density of wasps is modeled by the function $w(t)$, and the parameters
$\alpha$, $\beta$, $d$, and $g$ are positive constants. The fixed
points and linear stability of this particular system was explored in
(cite type II paper). 

For this situation, though, there is a distribution associated with
the butterflies' propensity to make use of the anti-pheromone. In this
case we do not model the male and female populations separately, but
treat them as a single population that intermingles in a relatively
uniform fashion. The propensity a butterfly to make use of the
anti-pheromone depends on a new parameter, $\theta$. The value of
$\theta$ is assume to vary between $0$ and some positive constant,
$L$, and the larger the value of $\theta$ the more likely an
individual is to make use of the anti-pheromone. The distribution of
the population of butterflies is now dependent on both the time and
the parameter,
\begin{eqnarray}
  b & = & b(t,\theta).
\end{eqnarray}

Our first task is to determine how the type II, or Holling type
response, should be expressed in this new context. Formal derivations
of the response function for the case given in equations
\ref{eq:initialSystem1} and \ref{eq:initialSystem2} are provided in
(cite formal derivation of type II). In the derivation here, however,
we follow the approach in Hollings original discussion (cite
Hollings). We begin by finding an expression that relates the rate of
butterfly predation by a single wasp,
\begin{eqnarray}
  \label{eq:processingTime}
  Y = ????
\end{eqnarray}

In this case $p(\theta)$ represents the impact of the use of the
anti-pheromone, and we assume a linear relationship with a positive
slope. As $\theta$ increases the more likely an individual butterfly
is to make use of the anti-pheromone and the more likely a wasp is to
locate and ride along with a butterfly. If we assume $p(0)$ is some
minimal value $p(0)=a$, and the relationship has a positive slope then
the normalized form is
\begin{eqnarray}
  \label{eq:linearFormP}
  p(\theta) & = & \frac{m}{something} \left( \theta + a \right).
\end{eqnarray}
where the slope is chosen so that the area under the line from
$\theta=0$ to $\theta=L$ is one.

Turning back to equation \ref{eq:processingTime}, the rate of
predation is solved for, and the result is
\begin{eqnarray}
  \label{eq:rateOfPredation}
  r & = &  \frac{p(\theta) b(t,\theta) }{1 + \gamma p(\theta) b(t,\theta)}.
\end{eqnarray}
This result can be immediately substituted into equation
\ref{eq:initialSystem1} in the more traditional form as
\begin{eqnarray}
  \label{eq:butterflyPredationRate}
  \beta \cdot w(t) \frac{p(\theta) b(t,\theta) }{c +  p(\theta) b(t,\theta)}
\end{eqnarray}
Equation \ref{eq:initialSystem2}, though, is for the rate of predation
for the single wasp population, so the impact for the wasps must be
accumulated across the entire butterfly population,
\begin{eqnarray}
  \label{eq:waspPredationRate}
  \int^L_{\theta=0} g \cdot w(t) \frac{p(\theta) b(t,\theta) }{c + p(\theta) b(t,\theta)} ~ d\theta.
\end{eqnarray}

We are almost ready to put these terms together for the current
model. It is assumed that the mixing of the butterfly population is
uniform, and a standard Fickian diffusion is assumed to approximate
the sharing of the genetic information relative to the propensity to
use the anti-aphrodisiac. The result is a coupled PDE and ODE system
given by 
\begin{eqnarray}
  \label{eq:odePDE1}
  \frac{\partial}{\partial t} b(t,\theta) & = &
      \alpha \cdot b(t,\theta) (K - b(t,\theta))
      - \beta \cdot w(t) \frac{p(\theta) b(t,\theta)}{c+p(\theta)b(t,\theta)}
      + \mu \frac{\partial^2}{\partial \theta^2} b(t,\theta) , \\
  \label{eq:odePDE2}
  \frac{d}{dt} w(t) & = & -d \cdot w(t) +
      \int^L_{\theta=0} g \cdot w(t) \frac{p(\theta) b(t,\theta) }{c + p(\theta) b(t,\theta)} ~ d\theta.
\end{eqnarray}

The system can be scaled as $b\rightarrow \bar{B}\hat{b}$,
$w\rightarrow \bar{W}\hat{w}$, $t\rightarrow \bar{T}\hat{t}$, and
$\theta\rightarrow \bar{\Theta}\hat{\theta}$. We choose $\bar{B}=$,
$\bar{W}=$, $\bar{T}=$, and $\bar{\Theta}=L$. The resulting system
can be reduced to the following form
\begin{eqnarray}
  \label{eq:scaledodePDE1}
  \frac{\partial}{\partial t} b & = &
      b (1 - b)
      -  w \frac{\hat{p}(\theta) b}{c+\hat{p}(\theta)b}
      + \mu \frac{\partial^2}{\partial \theta^2} b , \\
  \label{eq:scaledodePDE2}
  \frac{d}{dt} w & = & -d \cdot w +
      \int^1_{\theta=0} g \cdot w \frac{\hat{p}(\theta) b }{c + \hat{p}(\theta) b} ~ d\theta,
\end{eqnarray}
where
\begin{eqnarray}
  \hat{p}(\theta) & = & 1 + m \cdot \theta.
\end{eqnarray}


\subsection{Stability Analysis Of A Simplified System Of ODEs}

Before proceeding to the numerical approximation of the full model, we
first examine the stability of a simplified system. The simplified
system is found by examining an ODE with a similar form. By assuming
that the distribution of the butterflies is a constant consistent with
a single value of $\theta$, the system is approximated by the
following system of ODEs:
\begin{eqnarray}
  \label{eq:scaledODE1}
  \frac{d}{dt} b(t) & = &
      b(t) (1 - b(t))
      -  w(t) \frac{\hat{p}(\theta) b(t)}{c+\hat{p}(\theta)b(t)}, \\
  \label{eq:scaledODE2}
  \frac{d}{dt} w(t) & = & -d \cdot w(t) +
       g \cdot w(t) \frac{\hat{p}(\theta) b(t) }{c + \hat{p}(\theta) b(t)}.
\end{eqnarray}

by focusing on the system of ODEs, we can examine an approximation to
the system by treating the variable $\theta$ as a fixed parameter and
can gain some insight into the distributed system. The analysis here
focused on the stability of the linearized system, and we focus on the
non-trivial fixed points that occur in the first quadrant away from
$b=0$ and away from $w=0$.

\section{Numerical Approximation}

\section{Results}

\section{Conclusion}

\section{Acknowledgements}


\end{document}
