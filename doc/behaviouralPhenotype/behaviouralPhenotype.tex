\documentclass[12pt]{article}
\usepackage{graphicx}
\usepackage[margin=2cm]{geometry}

\title{A Continuous Model Of Behavioural Phenotype: An Example Of
  Interaction Between \textit{Pieris Brasicae} and
  \textit{Trichogramma} Wasps}

\author{Kelly Black\thanks{Coresponding Author, kjblack@gmail.con},
  \and Malcolm R Adams, \and Aladeen Al Basheer, \and Caner Kazanci,
  \and Bernard Patten, \and Stuart Whipple, \and Sofya Zaytseva}

\begin{document}

\maketitle

\section{Introduction}

The variation in animal behaviours as well as the variation in
genetics for a given species plays a role in the broader population
dynamics of the species as well as the other animals that interact
with the species. A variety of mathematical models have been developed
to explore this important aspect of interactions. The models include
compartment models, stochastic, agent based models, as well as
combinations of these approaches. One difficulty with such approaches
is the lack of analytic tools to examine the general properties of the
resulting models.

We propose, instead, an approach assuming a continuous distribution
associated with one behavioural aspect of an animal's behaviour. The
example chosen is based on the interactions beteen parasitic wasps,
\textit{Trichogramma} Wasps, and the butterfly \textit{Pieris
  Brasicae} (cite Huigens).  The propensity of \textit{Pieris
  Brasicae} to employ a chemical associated with mating behaviour is
modeled as a continuous distribution, and a distributed model is
developed to approximate the resulting interactions.

In this treatment we approximate the system as consisting of two
species. The parastic wasp requires a relatively long interaction
time, and a predator-prey relationship is assumed. In particular, the
predation rates are approximated using a type II response. Due to the
variation in behaviour, though, we adapt the standard Holling response
function for this situation.

We first provide a more detailed overview of behavioural phenotypes
including some previous efforts to model the phenomena. Next, the
model used to provide insight into the phenomena is developed
including a re-derivation of a type II response assuming a continuous
variation in one behaviour. Following the derivation of the model, the
numerical approximation to the distributed system is briefly
discussed. Finally, some results from our numerical explorations are
provided.

\section{Behavioural Phenotype}

A number of authors have noted the importance of the role of variation
in a species' genetics or habits (cite Pinter-Wollman, Kortet,
Keiser). We begin this discussion with the survey provided by Kortet,
\textit{et al} (cite). The authors note that within any large group of
animals their behaviours can vary. They go on to propose that the
diversity of behaviours can profoundly impact the broader dynamics of
the population.

Kortet, \textit{et al} (cite) note the impact of variation in the
distribution of behaviour, they also note potential challenges this
recognition implies. One difficulty is identifying and quantifying the
distribution. An example they provide is the methods used to capture
individuals in a population can provide a biased estimate as some
methods may be more or less likely to capture individuals based on how
bold or timid they may be.

(Find example from their paper.)  Other mathematical models have been
developed.... Compartmental models by citation. Other agent based
models by citation. Hybrid models by citation.

Another, detailed example of the importance of behavioural differences
between individuals in a population is provided by Keiser \textit{et
  al} (cite). In their work, the differences in levels of aggression
in a social spider, \textit{Stegodyphus dumicola} are
examined. Experiments in sampling individuals found significant
differences in boldness levels. They also found that the variation of
boldness levels within a group impacted the spread of microbes between
individuals.

A mathematical model for this system is provided by Pinter-Wollman,
\textit{et al} (cite). The focus of their work is in examining the
impact of different boldness levels of a key individual.  They note
that a small number of individuals within a colony may exhibit vastly
different levels of aggression, and a small number of individuals can
have a disproportionate impact in the broader success of the colony.

As a means to model this important aspect of a colony, Pinter-Wollman,
\textit{et al} provide a computational model consisting of an agent
based model with a stochastically generated population. Their approach
is based on a small number of rules that result in a distribution of
behaviours. The temporal interactions within the colony are
approximated via discrete time steps and stochastic decision
making. The external interactions are approximated using an
ordinary differential equation model.

In the resulting statistical analyses, Pinter-Wollman, \textit{et al}
found that different levels of boldness within key individuals
resulted in a difference in the mean boldness levels of a broader
group. For example, they found the differences in boldness impacted
the rate of capture of prey. They also found that the rules themselves
impacted also impacted the disease dynamics within the population.


\section{Modeling Genetic Variance As A Continuous Distribution}

\section{Numerical Approximation}

\section{Results}

\section{Conclusion}

\section{Acknowledgements}


\end{document}
