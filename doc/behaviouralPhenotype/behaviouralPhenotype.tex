\documentclass[12pt]{article}
\usepackage{graphicx}
\usepackage[margin=2cm]{geometry}

\title{A Continuous Model Of Behavioural Phenotype: An Example Of
  Interaction Between \textit{Pieris Brasicae} and
  \textit{Trichogramma} Wasps}

\author{Kelly Black\thanks{Coresponding Author, kjblack@gmail.con},
  \and Malcolm R Adams, \and Aladeen Al Basheer, \and Caner Kazanci,
  \and Bernard Patten, \and Stuart Whipple, \and Sofya Zaytseva}

\begin{document}

\maketitle

\section{Introduction}

The variation in animal behaviours as well as the variation in
genetics for a given species plays a role in the broader population
dynamics of the species as well as the other animals that interact
with the species. A variety of mathematical models have been developed
to explore this important aspect of interactions. The models include
compartment models, stochastic, agent based models, as well as
combinations of these approaches. One difficulty with such approaches
is the lack of analytic tools to examine the general properties of the
resulting models.

We propose, instead, an approach assuming a continuous distribution
associated with one behavioural aspect of an animal's behaviour. The
example chosen is based on the interactions beteen parasitic wasps,
\textit{Trichogramma} Wasps, and the butterfly \textit{Pieris
  Brasicae} (cite Huigens).  The propensity of \textit{Pieris
  Brasicae} to employ a chemical associated with mating behaviour is
modeled as a continuous distribution, and a distributed model is
developed to approximate the resulting interactions.

In this treatment we approximate the system as consisting of two
species. The parastic wasp requires a relatively long interaction
time, and a predator-prey relationship is assumed. In particular, the
predation rates are approximated using a type II response. Due to the
variation in behaviour, though, we adapt the standard Holling response
function for this situation.

We first provide a more detailed overview of behavioural phenotypes
including some previous efforts to model the phenomena. Next, the
model used to provide insight into the phenomena is developed
including a re-derivation of a type II response assuming a continuous
variation in one behaviour. Following the derivation of the model, the
numerical approximation to the distributed system is briefly
discussed. Finally, some results from our numerical explorations are
provided.

\section{Behavioural Phenotype}

\section{Modeling Genetic Variance As A Continuous Distribution}

\section{Numerical Approximation}

\section{Results}

\section{Conclusion}


\end{document}
